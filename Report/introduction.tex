Model verification techniques can be used to improve the quality of software. These techniques require the behaviour of the software to be modelled, we then want to check if the model behaves conforming to some requirement. Different languages are proposed and well studied to express these requirements, examples are LTL, CTL, CTL* and $\mu$-calculus. Once the behaviour is modelled and the requirement is expressed in some language we can use modal checking techniques to determine if the model satisfies the requirement.

These techniques are well suited to model and verify the behaviour of a single software product. However software systems can be designed to have certain parts enabled or disabled. This gives rise to many software products that all behave very similar but not identical, such a collection is often called a \textit{product family}. The differences between the products in a product family is called the \textit{variability} of the family. A family can be verified by using the above mentioned techniques to verify every single product independently. However this approach does not use the similarities in behaviour of these different products, an approach that would make use of the similarities could potentially be a lot more efficient.

\textit{Labelled transition systems} (LTSs) are often used to model the behaviour of a system, while it can model behaviour well it can't model variability. Efforts to also model variability include I/O automata, modal transition systems and \textit{featured transition systems} (FTSs). Specifically the latter is well suited to model all the different behaviours of the software products as well as the variability of the entire system in a single model.

Efforts have been made to verify requirements for entire FTSs, as well as to be able to reason about features. Notable contributions are fLTL, fCTL and fNuSMV. However, as far as we know, there is no technique to verify an FTS against a $\mu$-calculus formula. The modal $\mu$-calculus is very expressive and subsumes other temporal logics like LTL, CTL and CTL*. In this thesis we will introduce a technique to do this. We first look at LTSs, the modal $\mu$-calculus and FTSs. Next we will look at an existing technique to verify an LTS, namely solving \textit{parity games}, as well as show how this technique can be used to verify an FTS by verifying every software product it describes independently. An extension to this technique is than proposed, namely solving \textit{variability parity games}. We will formally define variability parity games and prove that solving a them can be used to verify FTSs.