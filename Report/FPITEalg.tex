
Using pessimistic games we can create an algorithm that solves the entire VPG incrementally. First we try to find $P_0$ and $P_1$ for all the configurations, next we partition the configuration set in two sets and try to improve $P_0$ and $P_1$ for these sets. We continue to do this until we have configuration sets of size $1$ and we simply solve the projection.

The pseudo code is presented in algorithm \ref{alg_IncPreSolve}, the algorithm relies on a \textsc{Solve} algorithm that solves a parity game. The \textsc{Solve} algorithm is some algorithm that solves parity games and can use the parameters $P_0$ and $P_1$ to more efficiently solve the parity game. The pessimistic parity games are not necessarily total so the \textsc{Solve} algorithm must be able to solve non-total games. When solving a parity game for which we have $P_0$ and $P_1$ we say that vertices in $P_0$ and $P_1$ are \textit{pre-solved}. 
\begin{algorithm}
	\caption{$\textsc{IncPreSolve}(G = (V,V_0,V_1, E, \Omega, \mathfrak{C}, \theta), P_0,P_1)$}\label{alg_IncPreSolve}
	\begin{algorithmic}[1]
		\If{$|\mathfrak{C}| = 1$}
		\State $\{c\} \gets \mathfrak{C}$
		\State $(W'_0,W'_1) \gets \textsc{Solve}(G_{|c}, P_0,P_1)$
		\State \Return $(\mathfrak{C} \times W'_0, \mathfrak{C} \times W'_1)$
		\EndIf
		\State $(P'_0,-) \gets \textsc{Solve}(G_{\triangleright0}, P_0, P_1)$
		\State $(-,P'_1) \gets \textsc{Solve}(G_{\triangleright1}, P_0, P_1)$
		\If{$P'_0 \cup P'_1 = V$}
		\State \Return $(\mathfrak{C} \times P'_0, \mathfrak{C} \times P'_1)$
		\EndIf
		\State $\mathfrak{C}^a, \mathfrak{C}^b \gets $ partition $\mathfrak{C}$ in non-empty parts
		\State $(W_0^a, W_1^a) \gets \textsc{IncPreSolve}(G \cap \mathfrak{C}^a, P'_0,P'_1)$
		\State $(W_0^b, W_1^b) \gets \textsc{IncPreSolve}(G \cap \mathfrak{C}^b, P'_0,P'_1)$
		\State $W_0 \gets W_0^a \cup W_0^b$
		\State $W_1 \gets W_1^a \cup W_1^b$
		\State \Return $(W_0,W_1)$
	\end{algorithmic}
\end{algorithm}

A \textsc{Solve} algorithm must correctly solve a game as long as the sets $P_0$ and $P_1$ are in fact vertices that are won by player $0$ and $1$ respectively. We prove that this is the case in the \textsc{IncPreSolve} algorithm.
\begin{theorem}
	Given VPG $\hat{G}$. For every $\textsc{Solve}(G,P_0,P_1)$ that is invoked during $\textsc{IncPreSolve}(\hat{G},\emptyset,\emptyset)$ we have winning sets $W_0,W_1$ for game $G$ for which the following holds:
	\[ P_0 \subseteq  W_0 \]
	\[ P_1 \subseteq  W_1 \]
	\begin{proof}
		When $P_0 = \emptyset$ and $P_1 = \emptyset$ the theorem holds trivially. So we will start the analyses after the first recursion. 
		
		After the first recursion the game is $\hat{G} \cap \mathfrak{X}$ with $\mathfrak{X}$ being either $\mathfrak{C}^a$ or $\mathfrak{C}^b$. The set $P_0$ is the winning set for player $0$ for game $\hat{G}_{\triangleright0}$ and the set $P_1$ is the winning set for player $1$ for game $\hat{G}_{\triangleright1}$. In the next recursion the game is $\hat{G} \cap \mathfrak{X} \cap \mathfrak{X}'$ with $P_0$ being the winning set for player $0$ in game $(\hat{G} \cap \mathfrak{X})_{\triangleright0}$ and $P_1$ being the winning set for player $1$ in game $(\hat{G} \cap \mathfrak{X})_{\triangleright1}$. After the first recursion the game is always of the form  $(\hat{G} \cap \mathfrak{X}^0 \cap \dots \cap \mathfrak{X}^{k-1}) \cap \mathfrak{X}^k$. Furthermore $P_0$ is the winning set for player $0$ for game $(\hat{G} \cap \mathfrak{X}^0 \cap \dots \cap \mathfrak{X}^{k-1})_{\triangleright0}$ and $P_1$ is the winning set for player $1$ for game $(\hat{G} \cap \mathfrak{X}^0 \cap \dots \cap \mathfrak{X}^{k-1})_{\triangleright1}$.
		
		Next we inspect the three places \textsc{Solve} is invoked:
		\begin{enumerate}
			\item Consider the case where there is only one configuration in $\mathfrak{C}$ (line 1-5). Because $P_0$ is the winning set for player $0$ for game $(\hat{G} \cap \mathfrak{X}^0 \cap \dots \cap \mathfrak{X}^{k-1})_{\triangleright0}$ the vertices in $P_0$ are won by player $0$ in game $G_{|c}$ for all $c \in \mathfrak{X}^0 \cap \dots \cap \mathfrak{X}^{k-1}$ (using theorem \ref{the_pess_is_winning_for_all_conf}). This includes the one element in $\mathfrak{C}$. So we can conclude $P_0 \subseteq W_0$ where $W_0$ is the winning set for player $0$ in game $G_{|c}$ where $\{c\} = \mathfrak{C}$.
			
			Similarly for player $1$ we can conclude $P_1 \subseteq W_1$ and the theorem holds in this case.
			\item On line $6$ the game $G_{\triangleright0}$ is solved with $P_0$ and $P_1$. Because $G = \hat{G} \cap \mathfrak{X}^0 \cap \dots \cap \mathfrak{X}^{k-1} \cap \mathfrak{X}^k$ and $P_0$ is the winning set for player $0$ for game $(\hat{G} \cap \mathfrak{X}^0 \cap \dots \cap \mathfrak{X}^{k-1})_{\triangleright0}$ and $P_1$ is the winning set for player $1$ for game $(\hat{G} \cap \mathfrak{X}^0 \cap \dots \cap \mathfrak{X}^{k-1})_{\triangleright1}$ we can apply lemma \ref{lem_pessimistic_subgames} to conclude that the theorem holds in this case.
			\item On line $7$ we apply the same reasoning and lemma to conclude that the theorem holds in this case.
		\end{enumerate}
	\end{proof}
\end{theorem}

Next we prove the correctness of the algorithm, assuming the correctness of the \textsc{Solve} algorithm.
\begin{theorem}
	%TODO hat notatie
	Given VPG $\hat{G} = (\hat{V},\hat{V}_0,\hat{V}_1,\hat{E},\hat{\Omega},\mathfrak{C},\theta)$ and $(W_0,W_1) = \textsc{IncPreSolve}(\hat{G},\emptyset,\emptyset)$. For every configuration $c \in \mathfrak{C}$ and winning sets $\hat{W}_0^c, \hat{W}_1^c$ for game $\hat{G}$ player for $c$ it holds that:
	\[ (c,v) \in W_0 \iff v \in \hat{W}_0^c \]
	\[ (c,v) \in W_1 \iff v \in \hat{W}_1^c \]
	\begin{proof}
		We will prove the theorem by applying induction on $\mathfrak{C}$.
		
		\textbf{Base} $|\mathfrak{C}| = 1$, when there is only one configuration, being $c$, then the algorithm solves game $G_{|c}$. The product of the winning sets and $\{c\}$ is returned, so the theorem holds.
		
		\textbf{Step} Consider $P_0'$ and $P_1'$ as calculated in the algorithm (line 6-7). By theorem \ref{the_pess_is_winning_for_all_conf} all vertices in $P_0'$ are won by player $0$ in game $G_{|c}$ for any $c \in \mathfrak{C}$, similarly for $P_1'$ and player $1$.
		
		If $P_0' \cup P_1' = V$ then the algorithm returns $(\mathfrak{C} \times P_0',\mathfrak{C} \times P_1')$. In which case the theorem holds because there are no configuration vertex combinations that are not in either winning set and theorem \ref{the_pess_is_winning_for_all_conf} proves the correctness.
		
		If $P_0' \cup P_1' \neq V$ then we have winning sets $(W_0^a, W_1^a)$ for which the theorem holds (by induction) for game $G \cap \mathfrak{C}^a$ and $(W_0^b, W_1^b)$ for which the theorem holds (by induction) for game $G \cap \mathfrak{C}^b$. The algorithm returns $(W_0^a \cup W_0^b, W_1^a \cup W_1^b)$. Since $\mathfrak{C}^a \cup \mathfrak{C}^b = \mathfrak{C}$ all vertex configuration combinations are in the winning sets and the correctness follows from induction.
	\end{proof}
\end{theorem}

\subsection{Fixed-point approximation algorithm}
Parity games can be solved by solving an alternating fixed point formula, as shown in \cite{WALUKIEWICZ2002311}. We will consider PG $G = (V,V_0,V_1, E, \Omega)$ with $d$ distinct priorities. We can apply \textit{priority compression} to make sure every priority in $G$ maps to a value in $\{0,\dots,d-1\}$ or $\{1, \dots, d\}$ \cite{SolvingInPractice,FPITE}. We assume without loss of generality that the priorities map to $\{0,\dots,d-1\}$ and that $d-1$ is even. 

Consider the following formula
\[ S(G = (V,V_0,V_1,E,\Omega)) = \nu Z_{d-1}. \mu Z_{d-2}. \dots . \nu Z_0. F_0(Z_{d-1},\dots,Z_0) \]
with
\[ F_0(Z_{d-1},\dots,Z_0) = \{ v \in V_0\ |\ \exists_{w\in V} (v,w) \in E \wedge Z_{\Omega(w)} \} \cup \{ v \in V_1\ |\ \forall_{w\in V} (v,w) \in E \implies Z_{\Omega(w)} \} \]
where $Z_i \subseteq V$. The formula $\nu X. f(X)$ solves the greatest fixed-point of $X$ in $f$, similarly $\mu X.f(X)$ solves the least fixed-point of $X$ in $f$.

To understand the formula we consider sub-formula $\nu Z_0. F_0(Z_{d-1},\dots,Z_0)$. This formula holds for vertices from which player $0$ can either force the play into a node with priority $i > 0$ for which $Z_i$ holds or the player can stay in vertices with priority $0$ indefinitely. The formula $\mu Z_0. F_0(Z_{d-1},\dots,Z_0)$ holds for vertices from which player $0$ can force the play into a node with priority $i > 0$ for which $Z_i$ holds in finitely many steps.

As shown in \cite{WALUKIEWICZ2002311}, solving $S(G)$ gives the winning set for player $0$ in game $G$. A concrete algorithm is introduced in \cite{FPITE}, note that this algorithm can solve finite games. We will extend this algorithm such that it calculates $S(G)$ in an efficient manner by using $P_0$ and $P_1$ where it is known beforehand that vertices in $P_0$ are won by player 0 and vertices in $P_1$ are won by player 1. 

\subsubsection{Fixed-point approximation}
As shown in \cite{Emerson:1986:MCP:900378} we can calculate fixed-point $\mu X.f(X)$ when $f$ is monotonic in $X$ by approximating $X$.
\[ \mu X.f(X) = \bigcup_{i \geq 0} X^i \]
where $X^i = f(X^{i-1})$ for $i > 0$ and $X^0 \subseteq \mu X.f(X)$. So picking the smallest value possible for $X_0$ will always correctly calculate $\mu X. f(X)$.

Similarly we can calculate fixed-point $\nu X.f(X)$ when $f$ is monotonic in $X$ by approximating $X$.
\[ \nu X.f(X) = \bigcap_{i \geq 0} X^i \]
where $X^i = f(X^{i-1})$ for $i > 0$ and $X^0 \supseteq \nu X.f(X)$. So picking the largest value possible for $X_0$ will always correctly calculate $\nu X. f(X)$.

Clearly the formula $F_0(Z_{d-1},\dots,Z_0)$ is monotonic in any $Z_j$, so we can calculate $Z_{d-1}$ by approximating every $Z_j$ starting at $V$ for $\nu Z_j$ and starting at $\emptyset$ for $\mu Z_j$.


\subsubsection{Pre-solved games}
Let $G$ be a PG and let sets $P_0$ and $P_1$ be such that vertices in $P_0$ are won by player $0$ and vertices in $P_1$ are won by player $1$. We can fixed-point approximate $S(G)$ to calculate $W_0$, we know that $W_0$ is bounded by $P_0$ and $P_1$, specifically we have
\[ P_0 \subseteq W_0 \subseteq V\backslash P_1\]
We can use this restriction to efficiently approximate $S(G)$. If no bounds are know and we would approximate fixed-point formula $\nu Z_{d-1}\dots$ then we would start at $Z_{d-1}^0 = V$ which is the largest value possible, however given the bounds we can start our approximations of greatest fixed-point variables at $V\backslash P_1$ and start our approximations of least fixed-point variables at $P_0$. The following lemma's and theorems prove this.
\begin{lemma}
	\label{lem_fixpoint_bounds_nu}
	Given
	\begin{itemize}
		\item A complete lattice $\langle 2^A, \subseteq \rangle$,
		\item monotonic function $f : 2^A \rightarrow 2^A$ and
		\item $R^\bot \subseteq A$ and $R^\top \subseteq A$ such that $R^\bot \subseteq \nu X. f(X) \subseteq R^\top$
	\end{itemize}
	we approximate $X$ by starting with $X^0 = R^\top$. For any $i \geq 0$ it holds that
	\[ R^\bot \subseteq f(X^i) \subseteq R^\top \]
	\begin{proof}
		Assume $R^\bot \supset f(X^i)$. By fixed-point approximation we have $\nu X.f(X) = \cap_{j\geq0} X^j$, so we find $R^\bot \supset \nu X.f(x)$ which is a contradiction so $R^\bot \subseteq f(X^i)$.
		
		Assume $f(X^i) \supset R^\top$. Because of monotonicity we find $X^i \subseteq R^\top$ and therefore $f(X^i) \supset R^\top \supseteq X^i$. Using the Knaster-Tarski theorem (\ref{the_knaster_tarski}) we can conclude that the greatest fixed-point of $f(X)$ is larger than $f(X^i)$, so we find $\nu X.f(X) \supset R^\top$ which is a contradiction so $f(X^i) \subseteq R^\top$.
	\end{proof}
\end{lemma}

\begin{lemma}
	\label{lem_fixpoint_bounds_mu}
	Given
	\begin{itemize}
		\item A complete lattice $\langle 2^A, \subseteq \rangle$,
		\item monotonic function $f : 2^A \rightarrow 2^A$ and
		\item $R^\bot \subseteq A$ and $R^\top \subseteq A$ such that $R^\bot \subseteq \mu X. f(X) \subseteq R^\top$
	\end{itemize}
	we approximate $X$ by starting with $X^0 = R^\bot$. For any $i \geq 0$ it holds that
	\[ R^\bot \subseteq f(X^i) \subseteq R^\top \]
\end{lemma}

\begin{theorem}
	\label{the_FPITE_starting}
	Given PG $G = (V,V_0,V_1,E,\Omega)$ with $P_0$ and $P_1$ such that vertices  in $P_0$ are won by player $0$ in game $G$ and vertices in $P_1$ are won by player $1$ in game $G$ we can approximate the fixed-point variables by starting at $P_0$ for least fixed-points and starting at $V \backslash P_1$ for greatest fixed-points.
	\begin{proof}
		Let $f(Z_{d-1}) = \mu Z_{d-2}\dots\nu Z_0.F_0(Z_{d-1},\dots,Z_0)$. Because $\nu Z_{d-1}.f(Z_{d-1})$ calculates $W_0$ we know $P_0 \subseteq \nu Z_{d-1}.f(Z_{d-1}) \subseteq V \backslash P_1$ so we can start the fixed-point approximation at $Z_{d-1}^0 = V\backslash P_1$. Using lemma \ref{lem_fixpoint_bounds_nu} we find for any $i \geq 0$ we have $P_0 \subseteq f(Z_{d-1}^i) \subseteq V \backslash P_1$.
		
		Let $g(Z_{d-2}) = \nu Z_{d-3} \dots \nu Z_0. F_0(Z_{d-1}^i,Z_{d-2},\dots,Z_0)$. We found $P_0 \subseteq f(Z_{d-1}^i) \subseteq V \backslash P_1$, therefore we have $P_0 \subseteq \mu Z_{d-2}.g(Z_{d-2})\subseteq V \backslash P_1$ so we can start the fixed-point approximation at $Z_{d-2}^0 = P_0$. Using lemma \ref{lem_fixpoint_bounds_mu} we find that for any $j \geq 0$ we have $P_0 \subseteq g(Z_{d-2}^j) \subseteq V\backslash P_1$.
		
		We can repeat this logic up until $Z_0$ to conclude that the theorem holds.
	\end{proof}
\end{theorem}
We can now take the fixed-point algorithm presented in \cite{FPITE} and modify it by starting at $P_0$ and $V \backslash P_1$. The pseudo code is presented in algorithm \ref{alg_FPITE}, its correctness follows from \cite{FPITE} and theorem \ref{the_FPITE_starting}.
\begin{algorithm}
	\caption{Fixed-point iteration with $P_0$ and $P_1$}
	\label{alg_FPITE}
	\begin{multicols}{2}
		\begin{algorithmic}[1]
			\Function{FPIter}{$G = (V, V_0, V_1, E, \Omega), P_0, P_1$}
				\For{$i \gets d-1,\dots,0$}
					\State $\textsc{Init}(i)$
				\EndFor
				\Repeat
					\State $Z_0'\gets Z_0$
					\State $Z_0 \gets \textsc{Diamond}() \cup \textsc{Box}()$
					\State $i \gets 0$
					\While{$Z_i=Z_i' \wedge i < d-1$}
						\State $i \gets i+1$
						\State $Z_i' \gets Z_i$
						\State $Z_i \gets Z_{i-1}$
						\State $\textsc{Init}(i-1)$
					\EndWhile
				\Until{$i = d-1 \wedge Z_{d-1} = Z_{d-1}'$}
				\State \Return $(Z_{d-1},V\backslash Z_{d-1})$
			\EndFunction
		\end{algorithmic}\bigskip\bigskip
		\begin{algorithmic}[1]
			\Function{Init}{$i$}
				\State $Z_i \gets P_0$ if $i$ is odd, $V\backslash P_1$ otherwise
			\EndFunction
		\end{algorithmic}\bigskip
		\begin{algorithmic}[1]
			\Function{Diamond}{}
				\State \Return $\{ v \in V_0\ |\ \exists_{w\in V} (v,w) \in E \wedge w \in Z_{\Omega(w)}\}$
			\EndFunction
		\end{algorithmic}\bigskip
		\begin{algorithmic}[1]
			\Function{Box}{}
			\State \Return $\{ v \in V_1\ |\ \forall_{w\in V} (v,w) \in E \implies w \in Z_{\Omega(w)}\}$
			\EndFunction
		\end{algorithmic}
	\end{multicols}
\end{algorithm}

This algorithm is appropriate to use as a \textsc{Solve} algorithm in the \textsc{IncPreSolve} since it solves parity games that are not necessarily total and uses $P_0$ and $P_1$.

\subsection{Running time}
We will consider the running time for solving VPG $G = (V,V_0,V_1,E,\Omega,\mathfrak{C},\theta)$ product based and family based. We will use $n$ to denote the number of vertices, $e$ the number of edges, $c$ the number of configurations and $d$ the number of distinct priorities.

The fixed-point iteration algorithm without $P_0$ and $P_1$ runs in $O(e*n^d)$ (\cite{FPITE}). We can use this algorithm to solve $G$ product based, ie. solve all the projections of $G$. This gives a time complexity of $O(c*e*n^d)$.

Next we consider the \textsc{IncPreSolve} algorithm for a family based approach, observe that in the worst case we have to split the set of configurations all the way down to individual configurations. We can consider the recursion as a tree where the leafs are individual configurations and at every internal node the set of configurations is split in two. Since in the worst case there are $c$ leaves, there are at most $c-1$ internal nodes. At every internal node the algorithm solves two games and at every leaf the algorithm solves 1 game, so we get $c + 2c - 2 = O(c)$ games that are being solved by \textsc{IncPreSolve}. In the worst case there are no similarities between the configuration in $G$ and at every iteration $P_0$ and $P_1$ are empty. In this case the \textsc{FPIte} algorithm behaves the same as the original algorithm and has a time complexity of $O(e*n^d)$, this gives an overall time complexity of $O(c*e*n^d)$ which is equal to a product based approach.