As discussed in the preliminaries PGs can be solved either globally or locally. Similar to PGs we can solve VPGs either globally or locally. When locally solving a VPG we determine for which configurations a certain vertex is won by player $0$ and for which configurations the vertex is won by player $1$. When globally solving a VPG we determine this for every vertex in the VPG.

When solving a VPG globally we might encounter significant differences in parts of the game or intermediate results between configurations that we perhaps do not encounter when solving it locally because we can terminate earlier. Therefore we hypothesize that the increase in performance between globally-collectively solving VPGs and locally-collectively solving VPGs is greater than the increase performance between globally-independently solving VPGs and locally-independently solving VPGs.

The algorithms we have seen thus far are global algorithms, in this section we introduce local variants for the PG algorithms we have seen: Zielonka's recursive algorithm and fixed-point iteration algorithm. Furthermore we introduce local variants for the VPG algorithms we have seen: the recursive algorithm for VPGs and the incremental pre-solve algorithm.

\section{Locally solving parity games}
The two parity game algorithms introduced in the preliminaries (Zielonka's recursive algorithm and the fixed-point iteration algorithm) can be turned into local variants. This local variant can be used to solve VPGs locally-independently.

\subsection{Zielonka's recursive algorithm local}
\label{sec:zlnk_org_local}
To make Zielonka's recursive algorithm a local algorithm, we use the property that vertices won by player $\overline{\alpha}$ in game $G \backslash A$ are also won by player $\overline{\alpha}$ in game $G$. This property is proven in the following lemma.
\begin{lemma}
	\label{lem_overlinealphawinner}
	In $\textsc{RecursivePG}(G)$ any vertex won by player $\overline{\alpha}$ in game $G\backslash A$ is also won by $\overline{\alpha}$ in game $G$.
	\begin{proof}
		We simply observe the behaviour of the algorithm and note that the algorithm returns $B$ as winning for player $\overline{\alpha}$, since $W'_{\overline{\alpha}} \subseteq B$ by definition of the attractor set we find that any vertex in $W'_{\overline{\alpha}}$ is in $W_{\overline{\alpha}}$. The correctness of \textsc{RecursivePG} follows from \cite{ZIELONKA1998135}, therefore the lemma holds.
	\end{proof}
\end{lemma}

We use this property to, in some cases, not go into the second recursion because we already found the vertex we are trying to solve locally. We extend the algorithm with two parameters, $v_0$ and $\Delta$. Vertex $v_0$ represents the vertex that we try to solve. Parameter $\Delta \subseteq \{0,1\}$ contains a set of players for which we try to find vertex $v_0$, if we find $v_0$ to be won by a player that is in $\Delta$ we are done for that recursion. However if we find vertex $v_0$ to be won by a player that is not in $\Delta$ we need to solve the entire game. The algorithm can make use of this to tell a recursive call it needs to find the vertex to be won by player $\overline{\alpha}$ in the subgame because then this vertex is also won by player $\overline{\alpha}$ in the game itself. If the vertex is not won by $\overline{\alpha}$ the entire game needs to be solved because the winning sets are required in the second recursion. Pseudo code for the algorithm is provided in algorithm \ref{alg_zlnk_local}.
\begin{algorithm}
	\caption{$\textsc{RecursivePGLocal}(\textit{PG } G = (V,V_0,V_1, E, \Omega),v_0,\Delta)$}
	\label{alg_zlnk_local}
	\begin{algorithmic}[1]
		\State $m \gets \min\{ \Omega(v)\ |\ v \in V\}$
		\State $h \gets\max\{ \Omega(v)\ |\ v \in V\}$
		\If{$h = m$ or $V = \emptyset$}
		\If{$h$ is even or $V = \emptyset$}
		\State \Return $(V,\emptyset)$
		\Else
		\State \Return $(\emptyset, V)$
		\EndIf
		\EndIf
		\State $\alpha \gets 0$ if $h$ is even and $1$ otherwise
		\State $U \gets \{v \in V\ |\ \Omega(v) = h\}$
		\State $A \gets \alpha\textit{-Attr}(G, U)$
		\If{$\overline{\alpha} \in \Delta$}
		\State $(W_0', W_1') \gets \textsc{RecursivePGLocal}(G \backslash A,v_0,\{\overline{\alpha}\})$
		\Else
		\State $(W_0', W_1') \gets \textsc{RecursivePGLocal}(G \backslash A,v_0,\emptyset)$
		\EndIf
		\If{$W_{\overline{\alpha}}' =\emptyset$}
		\State $W_\alpha \gets A \cup W_\alpha'$
		\State $W_{\overline{\alpha}} \gets \emptyset$
		\Else
		\If{$\overline{\alpha} \in \Delta \wedge v_0 \in W'_{\overline{\alpha}}$}
		\State $W_\alpha \gets \emptyset$
		\State $W_{\overline{\alpha}} \gets W'_{\overline{\alpha}}$
		\Else
		\State $B \gets \overline{\alpha}\textit{-Attr}(G,W_{\overline{\alpha}}')$
		\If{$\overline{\alpha} \in \Delta \wedge v_0 \in B$}
		\State $W_\alpha \gets \emptyset$
		\State $W_{\overline{\alpha}} \gets B$
		\Else
		\State $(W_0'', W_1'') \gets \textsc{RecursivePGLocal}(G \backslash B,v_0,\Delta)$
		\State $W_\alpha \gets W_\alpha''$
		\State $W_{\overline{\alpha}} \gets W_{\overline{\alpha}}'' \cup B$
		\EndIf
		\EndIf
		\EndIf
		\State \Return $(W_0, W_1)$
	\end{algorithmic}
\end{algorithm}

To prove the correctness we show that winning sets resulting from \textsc{RecursivePGLocal} are mutually exclusive and that vertex $v_0$ is in the correct winning set.

The following theorem proves that the winning sets are mutually exclusive.
\begin{theorem}
	Given total parity game $G = (V,V_0,V_1,E,\Omega)$, vertex $v_0$ (which is not necessarily in $V$) and $\Delta \subseteq \{0,1\}$. We have for sets $(W_0,W_1) = \textsc{RecursivePGLocal}(G,v_0,\Delta)$ that 
	$W_0 \cap W_1 = \emptyset$.
	\begin{proof}
		Proof by induction on $G$ with induction hypothesis:
		\[ W_0 \cap W_1 = \emptyset\text{ and }W_0 \cup W_1 \subseteq V \]
		
		\textbf{Base:} When $G$ is empty or there is only one priority then the algorithm returns $V$ and $\emptyset$ for winning sets so the IH holds.
		
		\textbf{Step:} For the first recursion we can apply induction to find that $W'_0 \cap W'_1 = \emptyset$ and $W'_0 \cup W'_1 \subseteq V \backslash A$. If $W'_{\overline{\alpha}} = \emptyset$ then the algorithm returns $A \cup W'_\alpha$ and $\emptyset$ which makes the IH true.
		
		If $W'_{\overline{\alpha}} \neq \emptyset$ then we consider two cases. First consider $\overline{\alpha} \in \Delta \wedge v_0 \in W'_{\overline{\alpha}}$, then the algorithm returns $\emptyset$ and $W'_{\overline{\alpha}}$ as winning sets, since $W'_{\overline{\alpha}} \subseteq V\backslash A \subseteq V$ the IH holds. Otherwise we calculate a value for $B$, by definition of the attractor set we get $B \subseteq V$, if $\overline{\alpha} \in \Delta \wedge v_0 \in B$ we return $\emptyset$ and $B$ so the IH holds. Otherwise we go into the second recursion where we find by induction that $W''_0 \cap W''_1 = \emptyset$ and $W''_0 \cup W''_1 \subseteq V \backslash B$. The algorithm returns $W''_\alpha$ and $W''_{\overline{\alpha}} \cup B$ as winning sets. Because $B \subseteq V$ we find $W''_0 \cup W''_1 \cup B \subseteq V$, we can also conclude that $W''_\alpha \cap B = \emptyset$ so the IH holds.
	\end{proof}
\end{theorem}

The next theorem proves that vertex $v_0$ is always in the correct winning set.
\begin{theorem}
	Given total parity game $G = (V,V_0,V_1,E,\Omega)$, vertex $v_0$ (which is not necessarily in $V$), $\Delta \subseteq \{0,1\}$ and winning sets $Q_0$ and $Q_1$ for game $G$. We have for sets $(W_0,W_1) = \textsc{RecursivePGLocal}(G,v_0,\Delta)$ that either or both of the following statements hold:
	\begin{enumerate}[(I)]
		\item For some $\beta \in \{0,1\}$ we have $v_0 \in W_\beta$, $v_0 \in Q_\beta$ and $\beta \in \Delta$.
		\item $W_0 = Q_0$ and $W_1 = Q_1$.
	\end{enumerate}
\begin{proof}
	To prove the correctness we compare the behaviour of \textsc{RecursivePGLocal} with the behaviour of the original algorithm \textsc{RecursivePG}, which we know solves a parity game globally. Note that solving $G$ globally is equal to statement (II).
	
	Prove by induction on $G$.
	
	\textbf{Base: } When there are no vertices in $V$ or there is only one priority then the local algorithm behaves the same as the original algorithm so statement (II) holds.
	
	
	\textbf{Step: } We start by distinguishing two cases, first assume that $v_0 \notin V$. Vertex $v_0$ is also not in any subgame of $G$. In the first recursion we know by induction that either or both of the statements are true for $G\backslash A$, since $v_0$ is not in game $G\backslash A$ only statement (II) can be true. So the first recursion returns the same result as the original algorithm would. The if statements guarding the second recursion are always false because $v_0 \notin B$ and $v_0 \notin W'_{\overline{\alpha}}$ and similarly as with game $G\backslash A$ the vertex $v_0$ is not in $G\backslash B$. So the second recursion also returns the same values as the original algorithm would. We can conclude that when $v_0 \notin V$ the local algorithm behaves the same as the original and therefore statement (II) holds.
	
	Now consider $v_0 \in V$. Again we distinguish two cases, starting with $\overline{\alpha} \notin \Delta$. In the first recursive call we get $\Delta = \emptyset$, by induction either or both statements hold for $G\backslash A$, however with $\Delta = \emptyset$ statement (II) must be true. Therefore the first recursive call behaves the same as the original algorithm. If $W'_{\overline{\alpha}} = \emptyset$ then the parity game is solved and statement (II) holds for game $G$. If $W'_{\overline{\alpha}} \neq \emptyset$ then the if statements guarding the second recursion are always false since $\overline{\alpha} \notin \Delta$ so the second recursive call is always performed. Up until this second recursive call the local algorithm behaves the same as the original algorithm. From the correctness of the original algorithm we can conclude that vertices won by some player in game $G\backslash B$ are also won by that player in game $G$. This holds true for the local algorithm in this case since it has behaved identical up to this point. By induction we know either or both statements are true, if statement (II) is true for $G\backslash B$ then the algorithm behaves the same as the original and statement (II) is also true for game $G$. If statement (I) is true for $G \backslash B$ and $v_0$ is won by player $\beta$ in $G\backslash B$ then we know $v_0$ is also won by player $\beta$ in game $G$ and thus statement (I) holds for $G$.
	
	Next consider $\overline{\alpha} \in \Delta$ (and $v_0 \in V$). The first recursive call is made with $\Delta = \{\overline{\alpha}\}$. If $W'_{\overline{\alpha}} = \emptyset$ then statement (II) must be true for game $G\backslash A$ and the algorithm behaves the same as the original in which case statement (II) is true for $G$. If $W'_{\overline{\alpha}} \neq \emptyset$ then we consider $v_0$ being in $W'_{\overline{\alpha}}$, by induction we can conclude that $v_0$ is indeed won by player $\overline{\alpha}$ in game $G\backslash A$. Furthermore from the correctness of the original algorithm we can conclude that any vertex won by player $\overline{\alpha}$ in game $G\backslash A$ is also won by player $\overline{\alpha}$ in game $G$. So returning winning sets $\emptyset$ and $W'_{\overline{\alpha}}$ make statement (I) true for game $G$.
	
	Consider $v_0$ not being in $W'_{\overline{\alpha}}$ but in $B$. We can conclude that statement (II) is true for subgame $G\backslash A$ so the algorithm behaves the same as the original algorithm up until this point. Using the original algorithm we can see that any vertex in $B$ is won by player $\overline{\alpha}$ in game $G$ so if we find $v_0$ in $B$ we can return $B$ for player $\overline{\alpha}$ which includes $v_0$ and $\emptyset$ for player $\alpha$ to make statement (I) true for game $G$.
	
	If $v_0$ is also not in $B$ then we still conclude that statement (II) is true for subgame $G\backslash A$ and that the algorithm has behaved the same as the original algorithm up until this point. From the correctness of the original algorithm we can conclude that vertices won by some player in game $G\backslash B$ are also won by that player in game $G$. This holds true in the local algorithm since it has behaved identical up to this point. By induction we know either or both statements are true, if statement (II) is true for $G\backslash B$ then the algorithm behaves the same as the original and statement (II) is also true for game $G$. If statement (I) is true for $G \backslash B$ and $v_0$ is won by player $\beta$ in $G\backslash B$ then we know $v_0$ is also won by player $\beta$ in game $G$ and thus statement (I) holds for $G$.
\end{proof}
\end{theorem}
Calling $\textsc{RecursivePGLocal}(G, v_0, \{0,1\})$ with $v_0$ in $G$ either solves the full game (statement (II)) or correctly puts $v_0$ in either winning set (statement (I)). In both cases $v_0$ is in the correct winning set and therefore it is not in the other winning set, so the game is solved locally.

The time complexity of the local variant is the same as the original algorithm: $O(e*n^d)$. If vertex $v_0$ is not winning for a player in $\Delta$ than the algorithm behaves the same as the original so its worst-case complexity is the same.

\subsubsection{A simpler idea} A simpler version of the algorithm is where there is no $\Delta$ variable and the algorithm simply terminates when $v_0$ is found to be won for player $\overline{\alpha}$. After all, vertices won by player $\overline{\alpha}$ in the subgame are also won by player $\overline{\alpha}$ in the game itself. However this idea is flawed because this property only extends one recursion deep. So during some recursion it is possible that the algorithm finds a vertex to be won by player $\overline{\alpha}$ for the subgame $G\backslash A$ and therefore also for game $G$, however game $G$ might be a subgame itself of some game $H$. The vertex found to be won by player $\overline{\alpha}$ in $G\backslash A$ and $G$ might not be won by player $\overline{\alpha}$ in game $H$.

We disprove the following conjecture to show that there indeed can be a vertex that is won by player $\overline{\alpha}$ at some point but be won by player $\alpha$ in a higher recursion level.
\begin{conjecture}[Disproven]
	For any $\textsc{RecursivePG}(G')$ that is invoked during $\textsc{RecursivePG}(G)$ it holds that any vertex $v \in W'_{\overline{\alpha}}$ is won by player $\overline{\alpha}$ in game $G$.
\begin{proof}[Counterexample]
 Consider the following parity game $G$:\\
	\begin{center}
		\begin{tikzpicture}[->]
			\tikzstyle{even} = [diamond,draw,minimum size=0.75cm]
			\tikzstyle{odd}  = [rectangle,draw,shape aspect=1,minimum size=0.75cm]
			
			\node[even,label=north:$v_1$] (v1) at (20,20) {1};
			\node[odd, label=north:$v_3$] (v3) at (22,20) {3};
			\node[odd, label=north:$v_2$] (v2) at (24,20) {2};
			
			\path (v1) edge[loop left] (v1);
			\path (v1) edge (v3);
			\path (v3) edge (v2);
			\path (v2) edge[loop right] (v2);
		\end{tikzpicture}
	\end{center}

	All vertices are won by player $0$ ($v_1$ plays to $v_3$, $v_3$ must play to $v_2$ and $v_2$ must play to itself therefore we always get an infinite path of $v_2$'s).
	
	We solve this game using \textsc{RecursivePG} and write down the values of relevant variables below. We use the tilde decoration to indicate values for variables in the first recursion:\\
	\begin{tabular}{|p{20cm}}
	$\textsc{RecursivePG}(G)$:\\
	$h=3$,$\alpha=1$\\
	$A =\{v_3\}$\\
	\begin{tabular}{|p{20cm}}
		$\textsc{RecursivePG}(G\backslash A)$:\\
		$\tilde{h}=2,\tilde{\alpha}=0$\\
		$\tilde{A}=\{v_2\}$\\
		\begin{tabular}{|p{20cm}}
			$\textsc{RecursivePG}(G\backslash A\backslash \tilde{A})$\\
		\end{tabular}
		$\tilde{W}'_0 = \emptyset$\\
		$\tilde{W}'_1 = \{v_1\} = \tilde{W}'_{\overline{\tilde{\alpha}}}$\\
		\textbf{Vertex $v_1$ is in $\tilde{W}'_{\overline{\tilde{\alpha}}}$ however in $G$ the vertex is won by player} $\tilde{\alpha}$.\\
		$\tilde{B} = \{v_1\}$\\
		\begin{tabular}{|p{20cm}}
			$\textsc{RecursivePG}(G\backslash A\backslash \tilde{B})$\\
		\end{tabular}
		$\tilde{W}''_0 = \{v_2\}, \tilde{W}''_1 = \emptyset$\\
	\end{tabular}
	$W'_0 = W'_{\overline{\alpha}} = \{v_2\}$\\
	$W'_1 = W'_\alpha = \{v_1\}$\\
	$B = V$\\
	$W_0 = V$
	\end{tabular}\\\\

This counterexample disproves the conjecture.
\end{proof}\end{conjecture}
An algorithm that simply stops when $v_0$ is found to be won for player $\overline{\alpha}$ requires the above conjecture to be true, since this is not the case such an algorithm is flawed. This problem is solved by using the $\Delta$ variable, as is used in the \textsc{RecursivePGLocal} algorithm.
\subsection{Fixed-point iteration local}
The fixed-point iteration algorithm can be modified to locally solve a game for vertex $v_0$ by distinguishing two cases:
\begin{enumerate}
	\item If $d-1$ is even then the outermost fixed-point variable is a greatest fixed-point variable. When at some point $v_0 \notin Z_{d-1}$ then we know $v_0$ is never won by player $0$ and we are done.
	\item If $d-1$ is odd then the outermost fixed-point variable is a least fixed-point variable. When at some point $v_0 \in Z_{d-1}$ then we know $v_0$ is won by player $0$ and we are done.
\end{enumerate}

If vertex $v_0$ is won by player 0 in the first case or won by player 1 in the second case then the algorithm never terminates early. So in the worst-case the local algorithm behaves the same as the global algorithm, therefore we have an identical time complexity, ie. $O(e*n^d)$.

\section{Locally solving variability parity games}
To solve a VPG locally we try to find out for which configurations vertex $\hat{v}_0$ is won by player $1$ and for which configurations it is won by player $0$. We consider the two collective VPG algorithms we have seen thus far and create local variants of them.
\subsection{Local recursive algorithm for variability parity games}
We can modify the recursive algorithm for VPGs into a local variant. In section \ref{sec:zlnk_org_local} we have seen a local variant of the Zielonka's recursive algorithm for PGs.

\begin{algorithm}
	\caption{$\textsc{RecursiveUVPGLocal}(\textit{PG } G = (\\
		V : \hat{V} \rightarrow 2^\mathfrak{C},\\
		\hat{V}_0 \subseteq \hat{V},\\
		\hat{V}_1 \subseteq \hat{V},\\
		E : \hat{E} \rightarrow 2^\mathfrak{C},\\
		\hat{\Omega} : \hat{V}\rightarrow \mathbb{N},\\
		v_0 \in \hat{V},\\
		\Delta))$}\label{alg_zlnk_UVPG_local}
	\begin{algorithmic}[1]
		\State $m \gets \min\{ \hat{\Omega}(\hat{v})\ |\ V(\hat{v}) \neq \emptyset \}$
		\State $h \gets \max\{ \hat{\Omega}(\hat{v})\ |\ V(\hat{v}) \neq \emptyset \}$
		\If{$h = m$ or $V = \lambda^\emptyset$}
		\If{$h$ is even or $V = \lambda^\emptyset$}
		\State \Return $(V,\lambda^\emptyset)$
		\Else
		\State \Return $(\lambda^\emptyset, V)$
		\EndIf
		\EndIf
		\State $\alpha \gets 0$ if $h$ is even and $1$ otherwise
		\State $U \gets \lambda^\emptyset$, $U(\hat{v}) \gets V(\hat{v})$ for all $\hat{v}$ with $\hat{\Omega}(\hat{v}) = h$
		\State $A \gets \alpha\textit{-FAttr}(G, U)$
		\If{$\overline{\alpha} \in \Delta$}
		\State $(W_0', W_1') \gets \textsc{RecursiveUVPGLocal}(G \backslash A,v_0, \{\overline{\alpha}\})$
		\Else
		\State $(W_0', W_1') \gets \textsc{RecursiveUVPGLocal}(G \backslash A,v_0, \emptyset)$
		\EndIf
		\If{$W_{\overline{\alpha}}' =\lambda^\emptyset$}
		\State $W_\alpha \gets A \cup W_\alpha'$
		\State $W_{\overline{\alpha}} \gets \lambda^\emptyset$
		\Else
		\State $(W'''_0, W'''_1) \gets (\lambda^\emptyset, \lambda^\emptyset)$
		\If{$\overline{\alpha} \in \Delta$}
		\State $W'''_0(\hat{v}) \gets W'_0(\hat{v}) \cap W'_{\overline{\alpha}}(v_0)$ for all $\hat{v} \in \hat{V}$
		\State $W'''_1(\hat{v}) \gets W'_1(\hat{v}) \cap W'_{\overline{\alpha}}(v_0)$ for all $\hat{v} \in \hat{V}$
		\State $G \gets G \cap (\mathfrak{C} \backslash W'_{\overline{\alpha}}(v_0))$
		\State $W'_0(\hat{v}) \gets W'_0(\hat{v}) \backslash W'_{\overline{\alpha}}(v_0)$ for all $\hat{v} \in \hat{V}$
		\State $W'_1(\hat{v}) \gets W'_1(\hat{v}) \backslash W'_{\overline{\alpha}}(v_0)$ for all $\hat{v} \in \hat{V}$
		\EndIf
		\State $B \gets \overline{\alpha}\textit{-FAttr}(G,W_{\overline{\alpha}}')$
		\If{$\overline{\alpha} \in \Delta$}
		\State $W'''_0(\hat{v}) \gets W'_0(\hat{v}) \cap B(v_0)$ for all $\hat{v} \in \hat{V}$
		\State $W'''_1(\hat{v}) \gets W'_1(\hat{v}) \cap B(v_0)$ for all $\hat{v} \in \hat{V}$
		\State $G \gets G \cap (\mathfrak{C} \backslash B(v_0))$
		\State $W'_0(\hat{v}) \gets W'_0(\hat{v}) \backslash B(v_0)$ for all $\hat{v} \in \hat{V}$
		\State $W'_1(\hat{v}) \gets W'_1(\hat{v}) \backslash B(v_0)$ for all $\hat{v} \in \hat{V}$
		\EndIf
		\State $(W_0'', W_1'') \gets \textsc{RecursiveUVPGLocal}(G \backslash B)$
		\State $W_\alpha \gets W_\alpha'' \cup W'''_{\alpha}$
		\State $W_{\overline{\alpha}} \gets W_{\overline{\alpha}}'' \cup B \cup W'''_{\overline{\alpha}}$
		\EndIf
		\State \Return $(W_0, W_1)$
	\end{algorithmic}
\end{algorithm}

\subsection{Local incremental pre-solve algorithm}
The incremental pre-solve algorithm is particularly appropriate for local solving because if we find ${v}_0$ in  $P_\alpha$ then we know that ${v}_0$ is won by player $\alpha$ for every configuration, therefore we are done for that particular recursion. This can potentially reduce the recursion depth of the algorithm and therefore reduce the number of (pessimistic) games solved.

Furthermore when there is only one configuration the incremental pre-solve algorithm solves the corresponding parity game. When taking a local approach it is sufficient to solve this PG locally. Note that the pessimistic games still must be solved globally to find as much assistance as possible for further recursions.

Clearly it could be the case that $v_0$ is not found in either $P_0$ or $P_1$ and is only solved when there is one configuration left. In such a case the local algorithm behaves the same as the global algorithm so we have an identical time complexity, ie. $O(c*e*n^d)$.