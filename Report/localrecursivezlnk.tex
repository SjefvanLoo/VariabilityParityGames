We can modify the recursive algorithm for VPGs into a local variant. In section \ref{sec:zlnk_org_local} we have seen a local variant of the Zielonka's recursive algorithm for PGs that uses $\Delta \subseteq \{0,1\}$ to indicate for which player we are trying to find the specific vertex.

When solving a VPG we have vertex $\hat{v}_0$, we try to find for every configuration $c$ which player wins $\hat{v}_0$. Consider VPG $G$ with configuration set $\mathfrak{C}$. Unified PG $G_{\downarrow}$ contains vertices $\mathfrak{C} \times \{\hat{v}_0\}$, if we find out the winning player for each of these vertices we have solved the VPG locally. So we are going to solve a unified PG locally, however instead of finding the winner of a single vertex we are finding the winners for a set of vertices, specifically vertices: $\mathfrak{C} \times \{\hat{v}_0\}$.

When we locally solve a parity game using the recursive algorithm we can sometimes not go into the second recursion because we already found the $\hat{v}_0$, however when locally solving a unified PG we might find $\hat{v}_0$ for some configuration but not for all. So when we find $\hat{v}_0$ to be won by player $\overline{\alpha} \in \Delta$ for configurations $C \subseteq \mathfrak{C}$ then we remove all vertices with configurations $C$ from the game, ie. we remove vertices $C \times \hat{V}$. For the remaining vertices we do go into the second recursion. Pseudo code is presented in algorithm \ref{alg_zlnk_collective_local}, we introduce function $\textsc{LocalConfs}$ which find the configurations for which we have found the local solution.

The algorithm uses definitions to reason about projections of games and sets to configuration(s). Previously we introduced a simple projection of a unified PG to a configuration which was possible because vertices in a unified PG consist of pairs of configurations and origin vertices. We define a similar projection for sets of vertices.
\begin{definition}
	Given set $X \subseteq (\mathfrak{C} \times \hat{V})$ we define the projection of $X$ to $c \in \mathfrak{C}$, denoted by $X_{|c}$, as
	\[ X_{|c} = \{ \hat{v}\ |\ (c,\hat{v}) \in X \}\]
\end{definition}
We also need to be able to reason about projections not only to a single vertex but to a group of vertices for which we introduce the following two definitions.
\begin{definition}
	Given set $X \subseteq (\mathfrak{C} \times \hat{V})$ we define the projection of $X$ to $C \subseteq \mathfrak{C}$, denoted by $X_{||C}$, as
	\[ X_{||C} = X \cap (C \times \hat{V}) \]
\end{definition}
\begin{definition}
Given set $X \subseteq (\mathfrak{C} \times \hat{V})$ we define the complementary projection of $X$ to $C \subseteq \mathfrak{C}$, denoted by $X_{|\backslash C}$, as
\[ X_{|\backslash C} = X \backslash (C \times \hat{V}) \]
\end{definition}

\begin{algorithm}
	\caption{$\textsc{RecursiveUPGLocal}(\textit{PG } G = (\\
		V \subseteq \mathfrak{C} \times \hat{V},\\
		\hat{V}_0 \subseteq \hat{V},\\
		\hat{V}_1 \subseteq \hat{V},\\
		E \subseteq (\mathfrak{C} \times \hat{V}) \times (\mathfrak{C} \times \hat{V}),\\
		\hat{\Omega} : \hat{V} \rightarrow \mathbb{N}),\\
		\hat{v}_0 \in \hat{V},\\
		\Delta \subseteq \{ 0,1\})$}
	\label{alg_zlnk_collective_local}
	\begin{multicols}{2}
		\begin{algorithmic}[1]
			\State $m \gets \min\{ \hat{\Omega}(\hat{v})\ |\ (c,\hat{v}) \in V\}$
			\State $h \gets\max\{ \hat{\Omega}(\hat{v})\ |\ (c,\hat{v}) \in V\}$
			\If{$h = m$ or $V = \emptyset$}
			\If{$h$ is even or $V = \emptyset$}
			\State \Return $(V,\emptyset)$
			\Else
			\State \Return $(\emptyset, V)$
			\EndIf
			\EndIf
			\State $\alpha \gets 0$ if $h$ is even and $1$ otherwise
			\State $U \gets \{(c,\hat{v}) \in V\ |\ \hat{\Omega}(\hat{v}) = h\}$
			\State $A \gets \alpha\textit{-Attr}(G, U)$
			\If{$\overline{\alpha} \in \Delta$}
			\State $(W_0', W_1') \gets \textsc{RecursiveUPGLocal}(G \backslash A,\hat{v}_0,\{\overline{\alpha}\})$
			\Else
			\State $(W_0', W_1') \gets \textsc{RecursiveUPGLocal}(G \backslash A,\hat{v}_0,\emptyset)$
			\EndIf
			\If{$W_{\overline{\alpha}}' =\emptyset$}
			\State $W_\alpha \gets A \cup W_\alpha'$
			\State $W_{\overline{\alpha}} \gets \emptyset$
			\Else
				\State $C_W \gets \textsc{LocalConfs}(W'_{\overline{\alpha}})$
				\If{$(W'_{\overline{\alpha}})_{|\backslash C_W} = \emptyset$}
					\State $W_{\overline{\alpha}} \gets W'_{\overline{\alpha}}$
					\State $W_\alpha \gets (W'_\alpha \cup A)_{|\backslash C_W}$
				\Else
				\State $B \gets \overline{\alpha}\textit{-Attr}(G, (W'_{\overline{\alpha}})_{|\backslash C_W})$
				\State $C_B \gets \textsc{LocalConfs}(B)$
				\State $(W_0'', W_1'') \gets \textsc{RecursiveUPGLocal}(G \backslash B\backslash V_{|| C_W \cup C_B}, \hat{v}_0, \Delta)$
				\State $W_\alpha \gets W''_\alpha$
				\State $W_{\overline{\alpha}} \gets W''_{\overline{\alpha}} \cup B \cup (W'_{\overline{\alpha}})_{||C_W}$
				\EndIf
			\EndIf
			\State \Return $(W_0, W_1)$
		\end{algorithmic}\bigskip\bigskip\bigskip\bigskip\bigskip\bigskip\bigskip\bigskip\bigskip\bigskip\bigskip\bigskip\bigskip\bigskip\bigskip\bigskip\bigskip\bigskip\bigskip\bigskip\bigskip\bigskip\bigskip\bigskip\bigskip\bigskip
		\begin{algorithmic}[1]
			\Function{LocalConfs}{$X \subseteq V$}
			\If{$\overline{\alpha} \in \Delta$}
				\State \Return $\{ c \in \mathfrak{C}\ |\ (c, \hat{v}_0) \in X \}$
			\Else
				\State \Return $\emptyset$
			\EndIf
			\EndFunction
		\end{algorithmic}
	\end{multicols}
\end{algorithm}
\begin{lemma}
	\label{lem_overlinealphawinner}
	In \textsc{RecursivePG} any vertex won by player $\overline{\alpha}$ in game $G\backslash A$ is also won by $\overline{\alpha}$ in game $G$.
	\begin{proof}
		We simply observe the behaviour of the algorithm and note that the algorithm returns $B$ as winning for player $\overline{\alpha}$, since $W'_{\overline{\alpha}} \subseteq B$ by definition of the attractor set we find that any vertex in $W'_{\overline{\alpha}}$ is in $W_{\overline{\alpha}}$. The correctness of \textsc{RecursivePG} follows from \cite{ZIELONKA1998135}, therefore the lemma holds.
	\end{proof}
\end{lemma}
In order to prove the correctness we repeat parts of the original proof given in \cite{ZIELONKA1998135}. We first introduce a few auxiliary definitions and lemma's from \cite{ZIELONKA1998135}.

Consider total PG $(V,V_0,V_1,E,\Omega)$ in the next few definition and lemma's.
\begin{definition}\cite{ZIELONKA1998135}
Set $X \subseteq V$ is an $\alpha$-trap in $G$ if and only if player $\overline{\alpha}$ can keep the play inside $X$.
\end{definition}
\begin{lemma}\label{lem_attr_is_trap}
	\cite{ZIELONKA1998135}
	Set $V\backslash \alpha\textit{-Attr}(G,X)$ is an $\alpha$-trap in $G$ for any non-empty $X \subseteq V$.
\end{lemma}
\begin{lemma}
	\label{lem_attr_trap_is_trap}\cite{ZIELONKA1998135}
	Let $X \subseteq V$ be an $\alpha$-trap in $G$. Then $\overline{\alpha}\textit{-Attr}(G,X)$ is also an $\alpha$-trap in $G$.
\end{lemma}
Finally we observe that a winning set $W_\alpha$ of PG $G$ is an $\overline{\alpha}$-trap in $G$, since if $\overline{\alpha}$ could play to $W_{\overline{\alpha}}$ from a vertex $v \in W_\alpha$ then he/she would do so and $v$ would be winning for $\overline{\alpha}$.

First we prove the completeness of the algorithm by showing that the winning sets are mutually exclusive.
\begin{theorem}Given:
	\begin{itemize}
		\item VPG $\hat{G} = (\hat{V}, \hat{V}_0,\hat{V}_1,\hat{E},\hat{\Omega},\mathfrak{C},\theta)$,
		\item origin vertex $\hat{v}_0$,
		\item total unified PG $G = (V, \hat{V}_0, \hat{V}_1,E,\hat{\Omega})$ that is a subgame of, or equal to, unified PG $\hat{G}_{\downarrow}$,
		\item a set of players $\Delta \subseteq \{0,1\}$ and
		\item winning sets $(W_0,W_1) = \textsc{RecursiveUPGLocal}(G,\hat{v}_0,\Delta)$
	\end{itemize}
	we have $W_0\cap W_1 = \emptyset$ and $W_0 \cup W_1 \subseteq V$.
	\begin{proof}
		Proof by induction on $G$.
		
		\textbf{Base}: If there is only one priority or there are no vertices than one of the winning sets is empty and the other is equal to $V$ so the theorem holds.
		
		\textbf{Step}: In the first recursion step we find by induction that $W'_0$ and $W'_1$ are mutually exclusive and that $W'_0 \cup W'_1 \subseteq V\backslash A$. We find that $W'_0 \cup W'_1 \cup A \subseteq V$. If $W'_{\overline{\alpha}} = \emptyset$ then we return $A\cup W'_{\overline{\alpha}}$ and $\emptyset$ as winning sets in which case the theorem holds.
		
		If $W'_{\overline{\alpha}} \neq \emptyset$ then we find a value for $C_W$. If $(W'_{\overline{\alpha}})_{|\backslash C_W} = \emptyset$ then $W'_{\overline{\alpha}}$ only contains vertices with a configuration that is in $C_W$. Winning set $W_\alpha$ gets vertices only with a configuration that is not in $C_W$ therefore the two are mutually exclusive. Furthermore since $W'_0 \cup W'_1 \cup A \subseteq V$ we find $W_{\overline{\alpha}} \cup W_\alpha \subseteq V$ and the theorem holds in this case.
		
		If $(W'_{\overline{\alpha}})_{|\backslash C_W} \neq \emptyset$ then we find a value for $B$. Since $B$ is the attractor set of $(W'_{\overline{\alpha}})_{|\backslash C_W}$ and edges don't cross configurations the set $B$ contains no vertices with a configuration that is in $C_W$. Clearly $B \subseteq V$. The recursion step gives $W''_0$ and $W''_1$ which are mutually and have $W''_0 \cup W''_1 \subseteq G\backslash B \backslash V_{||C_W\cup C_B}$ by induction. The algorithm returns these winning sets with $B$ and $(W'_{\overline{\alpha}})_{||C_W}$. Both $B$ and $(W'_{\overline{\alpha}})_{||C_W}$ don't share any vertices with each other nor with the winning sets $W''_0$ and $W''_1$. Moreover all $B$ and $(W'_{\overline{\alpha}})_{||C_W}$ are subsets of $V$ so we can conclude that the theorem holds.
	\end{proof}
\end{theorem}

Next we prove the soundness of the algorithm by showing that vertices $\{c\} \times \hat{v}_0$ end up in the correct winning set.
\begin{theorem}
	Given:
	\begin{itemize}
		\item VPG $\hat{G} = (\hat{V}, \hat{V}_0,\hat{V}_1,\hat{E},\hat{\Omega},\mathfrak{C},\theta)$,
		\item origin vertex $\hat{v}_0$,
		\item total unified PG $G = (V, \hat{V}_0, \hat{V}_1,E,\hat{\Omega})$ that is a subgame of, or equal to, unified PG $\hat{G}_{\downarrow}$,
		\item configuration $c\in \mathfrak{C}$,
		\item winning sets $Q_0$ and $Q_1$ for game $G_{|c}$,
		\item a set of players $\Delta \subseteq \{0,1\}$ and
		\item winning sets $(W_0, W_1) = \textsc{RecursiveUPGLocal}(G,\hat{v}_0,\Delta)$
	\end{itemize}
either or both of the following statements hold:
	\begin{enumerate}[(I)]
		\item For some $\beta \in \Delta$ we have $(c,\hat{v}_0) \in W_\beta$ and $\hat{v}_0 \in Q_\beta$.
		\item $(W_0)_{|c} = Q_0$ and $(W_1)_{|c} = Q_1$.
	\end{enumerate}
\begin{proof}
	First observe that statement (II) means that the projection of $G$ to $c$ is solved correctly, if $G$ is solved globally then statement (II) holds for any $c$.
	
	We prove the theorem by comparing the behaviour of \textsc{RecursiveUPGLocal} with the behaviour of the original algorithm (\textsc{RecursivePG}) which solves a unified PG globally.
	
	Prove by induction on $G$.
	
	\textbf{Base}: If $G$ has only one priority or no vertices then \textsc{RecursiveUPGLocal} behaves the same as the original algorithm so statement (II) holds.

	\textbf{Step}: The algorithm makes a number of case distinction. In order to prove the correctness we inspect all the different paths through the algorithm. We use a Fitch notation to keep an overview of the different case distinctions and improve readability.\\
	\begin{longtable}{|p{15.2cm}}
		Case $G_{|c}$ is empty.\\
		\hline
		Winning sets $(W_0)_{|c}$ and $(W_1)_{|c}$ are empty since the winning sets will be subsets of $V$ and $V_{|c}$ is empty. Clearly $Q_0$ and $Q_1$ are empty so the theorem holds.
	\end{longtable}
	\begin{longtable}{|p{15.2cm}}
		Case $G_{|c}$ is not empty.\\
		\hline
		\begin{tabular}{|p{15cm}}
			Case $(c,\hat{v}_0) \notin V$\\
			\hline
			Vertex $(c,\hat{v}_0)$ is also not in any subgame of $G$. By induction we can conclude that either or both statements are true for subgame $G\backslash A$ with configuration $c$ and sets $(W'_0,W'_1)$. Since $(c,\hat{v}_0)$ is not in the subgame statement (II) must hold.\\\\
			\begin{tabular}{|p{14.8cm}}
				Case $W'_{\overline{\alpha}} = \emptyset$\\
				\hline
				In this case the algorithm returns all vertices to be winning for player $\alpha$. By induction we found that indeed all vertices in game $(G\backslash A)_{|c}$ are won by player $\alpha$. This means that when we consider game $G_{|c}$ player $\overline{\alpha}$ must play the token away from vertices $V_{|c} \backslash A_{|c}$ to win. When the play goes to vertices in $A_{|c}$ then player $\alpha$ can make sure that the play goes to one of the vertices with the highest priority. After that the play can either go to vertices in $A_{|c}$ or vertices in $V_{|c} \backslash A_{|c}$. In the former case play will again go to a vertex with the highest priority. In the latter case player $\overline{\alpha}$ must play back to $A_{|c}$ or loose the game, however playing to $A_{|c}$ causes the play to go to the highest vertex again. Since the highest vertex has a parity equal to $\alpha$ every play is won by player $\alpha$ and we can conclude that indeed every vertex in $G_{|c}$ is won by player $\alpha$ and that statement (II) holds.
			\end{tabular}
			\begin{tabular}{|p{14.8cm}}
				Case $W'_{\overline{\alpha}} \neq \emptyset$\\
				\hline
				Since $(c,\hat{v}_0) \notin V$ we conclude $c \notin C_W$ and $((W'_{\overline{\alpha}})_{|\backslash C_W})_{|c} = (W'_{\overline{\alpha}})_{|c}$.\\\\
				\begin{tabular}{|p{14.6cm}}
					Case $(W'_{\overline{\alpha}})_{|\backslash C} = \emptyset$\\
					\hline
					All vertices in $V_{|c}$ are returned to be winning for player $\alpha$. We argued that the $C_W$ operations don't have an effect on the value of $(W'_{\overline{\alpha}})_{|c}$ therefore we can use the same argument as in case $W'_{\overline{\alpha}} = \emptyset$ to show that indeed all vertices in $V_{|c}$ are won by player $\alpha$ and that statement (II) holds.
				\end{tabular}
				\begin{tabular}{|p{14.6cm}}
					Case $(W'_{\overline{\alpha}})_{|\backslash C} \neq \emptyset$\\
					\hline
					Again we find that $c \notin C_B$. Set $V \backslash A$ is an $\alpha$-trap in $G$ (using lemma \ref{lem_attr_is_trap}), therefore $(V \backslash A)_{|c}$ is an $\alpha$-trap in $G_{|c}$. Set $(W'_{\overline{\alpha}})_{|c}$ is an $\alpha$-trap in $(G\backslash A)_{|c}$ (because it is a winning set). We can conclude that $(W'_{\overline{\alpha}})_{|c}$ is an $\alpha$-trap in $G_{|c}$. Using lemma \ref{lem_attr_trap_is_trap} we find that $B_{|c}$ is an $\alpha$-trap in $G_{|c}$, meaning that player $\overline{\alpha}$ can keep the play in $B_{|c}$, moreover player $\overline{\alpha}$ can play towards $(W'_{\overline{\alpha}})_{|c}$ where this player wins. We can conclude that any vertex in $B_{|c}$ is won by player $\overline{\alpha}$ and there are no edges that go from a vertex owned by player $\overline{\alpha}$ in $V_{|c} \backslash B_{|c}$ to $B_{|c}$. There might be such an edge owned by player $\alpha$, however if player $\alpha$ chooses plays along such an edge play ends up in $B_{|c}$ and player $\overline{\alpha}$ wins. Therefore we can simply solve game $G_{|c} \backslash B_{|c}$, the winning sets in this subgame are also winning in game $G_{|c}$.\\\\			
					The subgame created projected to $c$ is $G_{|c} \backslash B_{|c}$, since $\hat{v}_0$ is not in $G_{|c}$ it is also not in $G_{|c} \backslash B_{|c}$. By induction we find that statement (II) holds for $G_{|c} \backslash B_{|c}$ with sets $W''_0,W''_1$. The algorithm returns $(W''_\alpha)_{|c}$ for player $\alpha$ and $(W''_{\overline{\alpha}})_{|c} \cup B_{|c}$ for player $\overline{\alpha}$, this correctly solves game $G_{|c}$ therefore statement (II) holds.
				\end{tabular}
			\end{tabular}the correctness we inspect all the different paths through the algorithm. We use a 
		\end{tabular}\\
		\begin{tabular}{|p{15cm}}
			Case $(c,\hat{v}_0) \in V$\\
			\hline
			\begin{tabular}{|p{14.8cm}}
				Case $\overline{\alpha} \notin \Delta$\\
				\hline
				In the first recursion we get $\Delta = \emptyset$, therefore by induction we conclude that statement (II) holds for game $G\backslash A$ with configuration $c$ and sets $(W'_0, W'_1)$.\\\\
				\begin{tabular}{|p{14.6cm}}
					Case $W'_{\overline{\alpha}} = \emptyset$\\
					\hline
					As argued before if $(G \backslash A)_{|c}$ is entirely won by player $\alpha$ then $G_{|c}$ is entirely won by player $\alpha$. The algorithm returns all vertices to be winning for player $\alpha$ so statement (II) holds in this case.
				\end{tabular} 
				\begin{tabular}{|p{14.6cm}}
					Case $W'_{\overline{\alpha}} \neq \emptyset$\\
					\hline
					Since $\overline{\alpha} \notin \Delta$ \textsc{LocalConfs} will always return $\emptyset$. So $c \notin C_W$.\\\\
					
					\begin{tabular}{|p{14.4cm}}
						Case $(W'_{\overline{\alpha}})_{|\backslash C_W} = \emptyset$\\
						\hline
						Similarly, in this case $(G\backslash A)_{|c}$ is entirely won by player $\alpha$ and therefore statement (II) holds.
					\end{tabular}
					\begin{tabular}{|p{14.4cm}}
						Case $(W'_{\overline{\alpha}})_{|\backslash C_W} \neq \emptyset$\\
						\hline
						As argued before when $c \notin C_W$ and $c \notin C_B$ then if a vertex is won by a certain player in subgame $(G\backslash B)_{|c}$ it is also won by that player in game $G_{|c}$.\\
						
						Therefore we can conclude that if statement (I) holds for the subgame it holds for the game, similarly if statement (II) holds for the subgame it holds for the game. By induction at least one holds so the theorem holds in this case.
					\end{tabular}
				\end{tabular}
			\end{tabular}
			\begin{tabular}{|p{14.8cm}}
				Case $\overline{\alpha} \in \Delta$\\
				\hline
				The first recursion is called with $\Delta = \{\overline{\alpha}\}$. By induction we find that either or both statements hold for game $G\backslash A$, configuration $c$ and sets $(W'_0, W'_1)$.
				\begin{tabular}{|p{14.6cm}}
					Case $W'_{\overline{\alpha}} = \emptyset$\\
					\hline
					In this case statement (II) must hold. And similar as before we find that indeed all vertices in $G_{|c}$ are won by player $\alpha$.
				\end{tabular}
				\begin{tabular}{|p{14.6cm}}
					Case $W'_{\overline{\alpha}} \neq \emptyset$\\
					\hline
					\begin{tabular}{|p{14.4cm}}
						Case $c \in C_W$\\
						\hline
						Using lemma \ref{lem_overlinealphawinner} we find that anything won in subgame $G\backslash A$ by player $\overline{\alpha}$ is also won by that player in game $G$. We find $(c,\hat{v}_0)$ to be won by player $\overline{\alpha}$ in game $G\backslash A$ so this vertex is also won by player $\overline{\alpha}$ in game $G$. The algorithm returns $W'_{\overline{\alpha}}$ or $(W'_{\overline{\alpha}})_{||C_W}$ to be won by player $\overline{\alpha}$, this includes $(c,\hat{v}_0)$ so statement (I) holds.
					\end{tabular}
					\begin{tabular}{|p{14.4cm}}
						Case $c \notin C_W$\\
						\hline
						Statement (II) must hold for game $G \backslash A$ with configuration $c$ and sets $(W'_0,W'_1)$ because if statement (I) would hold we would have had $c \in C_W$.\\\\
						\begin{tabular}{|p{14.2cm}}
							Case $(W'_{\overline{\alpha}})_{|\backslash C_W} = \emptyset$\\
							\hline
							Again statement (II) must hold and we find that game $(G \backslash A)_{|c}$ is entirely won by player $\alpha$ and therefore game $G_{|c}$ is entirely won by player $\alpha$ so returning all vertices with configuration $c$ to be winning for player $\alpha$ makes statement (II) true.
						\end{tabular}
						\begin{tabular}{|p{14.2cm}}
							Case $(W'_{\overline{\alpha}})_{|\backslash C_W} \neq \emptyset$\\
							\hline
							\begin{tabular}{|p{14cm}}
								Case $c \in C_B$\\
								\hline
								Anything won by player $\overline{\alpha}$ in game $G\backslash A$ is also won by that player in game $G$. Set $B$ contains vertices such that player $\overline{\alpha}$ can force the play to a vertex that is won by player $\overline{\alpha}$ in game $G$, therefore anything in $B$ is also won by player $\overline{\alpha}$ in game $G$. The algorithm returns everything in $B$ to be winning for player $\overline{\alpha}$, so statement (I) holds.
							\end{tabular}
							\begin{tabular}{|p{14cm}}
								Case $c \notin C_B$\\
								\hline
								As argued before when $c \notin C_W$ and $c \notin C_B$ then if a vertex is won by a certain player in subgame $(G\backslash B)_{|c}$ it is also won by that player in game $G_{|c}$.\\
								
								Therefore we can conclude that if statement (I) holds for the subgame it holds for the game, similarly if statement (II) holds for the subgame it holds for the game. By induction at least one holds so the theorem holds in this case.
							\end{tabular}
						\end{tabular}
					\end{tabular}
				\end{tabular}
			\end{tabular}
		\end{tabular}
	\end{longtable}
\end{proof}
\end{theorem}

Using these two theorems we can conclude that $\textsc{RecursiveUPGLocal}(\hat{G}_{\downarrow},\hat{v}_0,\{0,1\})$ correctly locally solves VPG $\hat{G}$ for vertex $\hat{v}_0$.