The algorithms proposed to solve VPGs collectively have the same or a worse time complexity than the independent approach. The aim of the collective algorithms is to solve VPGs effectively when there are a lot of commonalities between configurations. A worst-case time complexity analyses does not say much about the performance in case there are many commonalities. In order to evaluate actual running time the algorithms are implemented and a number of test VPGs are created to test the performance on. In this section we discuss the implementation and look at the results.

During the previous sections we put forth a number of hypotheses about the performance of the algorithms introduced. In this section we evaluate these hypotheses, specifically we hypothesised
\begin{itemize}
	\item that the recursive algorithm for VPGs can attract a large number of configurations per origin vertex at the same time,
	\item that the recursive symbolic algorithm for VPGs performs well when solving VPGs originating from FTSs and
	\item that the increase in performance between a global-collective and local-collective approach is greater than the increase in performance between a global-independent and local-independent approach.
\end{itemize}