We explore some preliminary concepts relevant to solving algorithms.
\subsubsection{Set representation}
A set can straightforwardly be represented by a collection containing all the elements that are in the set. We call this an \textit{explicit} representation of a set. We can also represent sets \textit{symbolically} in which case the set of elements is represented by some sort of formula. A typical way to represent a set symbolically is through a boolean formula encoded in a \textit{binary decision diagram} \cite{BDD_book,Handbook_BDD_Chapter}. For example the set $S = \{0,1,2,4,5,7 \}$ can be expressed by boolean formula:
\[ F(x_2,x_1,x_0) = (x_2 \vee \neg x_1 \vee \neg x_0) \wedge (\neg x_2 \vee \neg x_1 \vee x_0) \]
where $x_0,x_1$ and $x_2$ are boolean variables. The formula gives the following truth table:\\
\begin{center}
	\begin{tabular}{|c|c|}
		\hline 
		$\mathbf{x_2x_1x_0}$ & $\mathbf{F(x_2,x_1,x_0)}$ \\ 
		\hline 
		000 & 1 \\ 
		\hline 
		001 & 1 \\ 
		\hline 
		010 & 1 \\ 
		\hline 
		011 & 0 \\ 
		\hline 
		100 & 1 \\ 
		\hline 
		101 & 1 \\ 
		\hline 
		110 & 0 \\ 
		\hline 
		111 & 1 \\ 
		\hline 
	\end{tabular} 
\end{center}
The function $F$ defines set $S'$ in the following way: $S' = \{x_2x_1x_0\ |\ F(x_2,x_1,x_0) = 1 \}$. As we can see set $S'$ and $S$ represent the same numbers. We can perform set operations, such as the union of two sets, on sets represented as boolean functions by performing the logical and operation on the functions.

We can represent boolean functions efficiently in BDDs, for a comprehensive treatment of BDDs we refer to \cite{BDD_book,Handbook_BDD_Chapter}. We will note here that given $x$ boolean variables and two boolean functions encoded as BDDs we can perform binary operations $\vee,\wedge,\neg$ on them in $O(2^{2x})=O(n^2)$ where $n = 2^x$ is the maximum set size that can be represented by $x$ variables \cite{BDD_running_time,Handbook_BDD_Chapter}.