\subsection{Set representation}
A set can straightforwardly be represented by a collection containing all the elements that are in the set. We call this an \textit{explicit} representation of a set. We can also represent sets \textit{symbolically} in which case the set of elements is represented by some sort of formula. A typical way to represent a set symbolically is through a boolean formula encoded in a \textit{binary decision diagram} \cite{BDD_book,Handbook_BDD_Chapter}. For example the set $S = \{0,1,2,4,5,7 \}$ can be expressed by boolean formula:
\[ F(x_2,x_1,x_0) = (x_2 \vee \neg x_1 \vee \neg x_0) \wedge (\neg x_2 \vee \neg x_1 \vee x_0) \]
where $x_0,x_1$ and $x_2$ are boolean variables. The formula gives the following truth table:\\
\begin{center}
	\begin{tabular}{|c|c|}
		\hline 
		$\mathbf{x_2x_1x_0}$ & $\mathbf{F(x_2,x_1,x_0)}$ \\ 
		\hline 
		000 & 1 \\ 
		\hline 
		001 & 1 \\ 
		\hline 
		010 & 1 \\ 
		\hline 
		011 & 0 \\ 
		\hline 
		100 & 1 \\ 
		\hline 
		101 & 1 \\ 
		\hline 
		110 & 0 \\ 
		\hline 
		111 & 1 \\ 
		\hline 
	\end{tabular} 
\end{center}
The function $F$ defines set $S'$ in the following way: $S' = \{x_2x_1x_0\ |\ F(x_2,x_1,x_0) = 1 \}$. As we can see set $S'$ and $S$ represent the same numbers. We can perform set operations on sets represented as boolean functions by performing logical operations on the functions. For example, given boolean formula's $f$ and $g$ representing sets $V$ and $W$ the formula $f \wedge g$ represents set $V \cap W$.

Boolean functions can efficiently be represented in BDDs, for a comprehensive treatment of BDDs we refer to \cite{BDD_book,Handbook_BDD_Chapter}. We will note here that given $x$ boolean variables and two boolean functions encoded as BDDs we can perform binary operations $\vee,\wedge$ on them in $O(2^{2x})=O(n^2)$ where $n = 2^x$ is the maximum set size that can be represented by $x$ variables \cite{BDD_running_time,Handbook_BDD_Chapter}. The running time specifically depends on the size of the decision diagrams, in general if the boolean functions are simple then the size of the decision diagram is also small and operations can be performed quickly.
\subsubsection{Symbolically representing sets of configurations}
For VPGs originating from an FTS the configuration sets are already boolean functions over the features. The formula's guarding the edges in the VPG will generally have relatively simple boolean functions, therefore they are specifically appropriate to represent as BDDs.

A set operation over two explicit sets can be performed in $O(n)$ where $n$ is the maximum size of the sets, this is better than the time complexity of a set operation using BDDs ($O(n^2)$). However if the BDDs are small then the set size can still be large but the set operations can be performed very quickly. This is a trade-off between worst case running time complexity and average actual running time; using a symbolic representation might yield better results if the sets are structured in such a way that the BDDs are small, however its worse case running time complexity will be worse.