We can consider a VPG as a PG that is the union of all its projections. We call the resulting PG the \textit{unification} of the VPG.
\begin{definition}
Given VPG $\hat{G} = (\hat{V},\hat{V}_0,\hat{V}_1, \hat{E},\hat{\Omega}, \mathfrak{C},\theta)$ we define the unification of $\hat{G}$, denoted as $\hat{G}_{\downarrow}$, as
\[  \hat{G}_{\downarrow} = \bigcup_{c\in \mathfrak{C}}\hat{G}_{|c} \]
where the union of two PGs is trivially defined as
\[ (V,V_0,V_1,E,\Omega) \cup (V',V_0',V_1',E',\Omega') = (V \cup V', V_0 \cup V_0', V_1 \cup V_1', E \cup E', \Omega \cup \Omega') \]
\end{definition}
We will use the hat decoration ($\hat{G},\hat{V},\hat{E},\hat{\Omega},\hat{W}$) when referring to a VPG and use no hat decoration when referring to a PG.

Every vertex in game $\hat{G}_{\downarrow}$ originates from a configuration and an original vertex. Therefore we can consider every vertex in a unification as a pair consisting of a vertex and a configuration, ie. $V = \mathfrak{C} \times \hat{V}$. We can consider edges in a unification similarly, ie. $E \subseteq (\mathfrak{C} \times \hat{V}) \times (\mathfrak{C} \times \hat{V})$. Note that edges don't cross configurations, ie. for every $((c,\hat{v}) , (c',\hat{v}')) \in E$ we have $c = c'$.

If we solve the PG that is the unification of a VPG we have solved the VPG, as shown in the next theorem .
\begin{theorem}
	\label{theA_solve_UVPG_is_solve_VPG}
	Given VPG $\hat{G} =  (\hat{V},\hat{V}_0,\hat{V}_1, \hat{E},\hat{\Omega}, \mathfrak{C},\theta)$. For winning sets $W_0$ and $W_1$ for game $\hat{G}_{\downarrow}$ and winning sets $\hat{W}^c_0$ and $\hat{W}^c_1$ for some configuration $c \in \mathfrak{C}$ it holds that
	\[(c,\hat{v}) \in W_\alpha \iff \hat{v} \in \hat{W}^c_\alpha  \text{, for }\alpha \in \{0,1\}  \]
	\begin{proof}
		The bi-implication is equal to  the following to implications.
		\[ (c,\hat{v}) \in W_\alpha \implies \hat{v} \in \hat{W}^c_\alpha  \text{, for }\alpha \in \{0,1\} \]
		and
		\[ (c,\hat{v}) \notin W_\alpha\implies \hat{v} \notin \hat{W}^c_\alpha \text{, for }\alpha \in \{0,1\}  \]
		
		Since the winning sets partition the game we have $\hat{v} \notin \hat{W}^c_\alpha \implies \hat{v} \in \hat{W}^c_{\overline{\alpha}}$ (similar for set $W$). Therefore it is sufficient to prove the first implication.
		
		Let $(c,\hat{v}) \in W_\alpha$, player $\alpha$ has a strategy to win game $\hat{G}_{\downarrow}$ from vertex $(c,\hat{v})$. Since $\hat{G}_{\downarrow}$ is the union of all the projections of $\hat{G}$ we can apply the same strategy to game $\hat{G}_{|c}$ to win vertex $\hat{v}$ as player $\alpha$. Because we can win $\hat{v}$ in the projection of $\hat{G}$ to $c$ we have $\hat{v} \in \hat{W}^c_\alpha$.
	\end{proof}
\end{theorem}

One of the properties of a PG is its totality; a game is total if every vertex has at least 1 outgoing vertex. Plays in a total PG will always result in an infinite path. VPGs are total, meaning that every vertex has, for every configuration $c \in \mathfrak{C}$ at least 1 outgoing vertex admitting $c$. Because VPGs are total a unified VPG is also total. For unified VPGs we can however further investigate its totality by introducing the notion of being total for every configuration. 
\begin{definition}
	A unified VPG $(V,V_0,V_1,E,\Omega)$ is total for every configuration iff for every $(c,v) \in V$ there exists a $(c,v') \in V$ such that $((c,v),(c,v')) \in E$.
\end{definition}

It follows that if a unified VPG is total then it is also total for every configuration as shown in the following lemma.
\begin{lemma}
	\label{lem_UVPG_total}
	A unified VPG $G = (V,V_0,V_1,E,\Omega)$ that is total is also total for every configuration.
	\begin{proof}
		Since $G$ is total we have for every $(c,v) \in V$ that there exists a $(c,v') \in V$ such that $((c,v),(c',v')) \in E$. Because unified VPGs don't have edges cross configurations we find that $c = c'$ and therefore $G$ is total for every configuration.
	\end{proof}
\end{lemma}

\subsubsection{Representing unified variability parity games}
Unified VPGs have a specific structure because they are the union of parity games that have the same vertices with the same owner and priority.

We can represent a set $X \subseteq (\mathfrak{C} \times \hat{V})$ as a complete function $f : \hat{V} \rightarrow 2^\mathfrak{C}$. The set $X$ and function $f$ are equivalent, denoted by the operator $=_\lambda$, iff the following relation holds:
\[ (c,\hat{v}) \in X \iff c \in f(\hat{v}) \]
We can also represent edges as a complete function $f : \hat{E} \rightarrow 2^\mathfrak{C}$. The set $E$ and function $f$ are equivalent, denoted by the operator $=_\lambda$, iff the following relation holds:
\[ ((c,\hat{v}),(c,\hat{v}')) \in E \iff c \in f(\hat{v},\hat{v}') \]
We write $\lambda^\emptyset$ to denote the function that maps every element to $\emptyset$, clearly $\lambda^\emptyset =_\lambda \emptyset$. We call using a set of pairs to represent vertices a \textit{set-wise} representation and using functions to represent vertices a \textit{function-wise} representation.


Finally we can simplify the priority function of a unified VPG, we don't actually need to create a new function that is the unification of all the projections, we can simply use the original priority assignment function because the following relation holds:
\[ \hat{\Omega}(c,\hat{v}) = \Omega(\hat{v}) \]