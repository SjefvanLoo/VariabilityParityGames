We can create a PG from a VPG by taking all the projections of the VPG, which are PGs, and combining them into one PG by taking the union of them. We call the resulting PG the \textit{unification} of the VPG. A parity game that is the result of a unification is called a \textit{unified PG}, also any total subgame of it will be called a unified PG. A unified PG always has a VPG from which it originated.
\begin{definition}
Given VPG $\hat{G} = (\hat{V},\hat{V}_0,\hat{V}_1, \hat{E},\hat{\Omega}, \mathfrak{C},\theta)$ we define the unification of $\hat{G}$, denoted as $\hat{G}_{\downarrow}$, as
\[  \hat{G}_{\downarrow} = \bigcup_{c\in \mathfrak{C}}\hat{G}_{|c} \]
where the union of two PGs is defined as
\[ (V,V_0,V_1,E,\Omega) \cup (V',V_0',V_1',E',\Omega') = (V \uplus V', V_0 \uplus V_0', V_1 \uplus V_1', E \uplus E', \Omega \uplus \Omega') \]
\end{definition}
We will use the hat decoration ($\hat{G},\hat{V},\hat{E},\hat{\Omega},\hat{W}$) when referring to a VPG and use no hat decoration when referring to a PG.

Every vertex in game $\hat{G}_{\downarrow}$ originates from a configuration and an original vertex. Therefore we can consider every vertex in a unification as a pair consisting of a vertex and a configuration, ie. $V = \mathfrak{C} \times \hat{V}$. We can consider edges in a unification similarly, so $E \subseteq (\mathfrak{C} \times \hat{V}) \times (\mathfrak{C} \times \hat{V})$. Note that edges don't cross configurations, ie. for every $((c,\hat{v}) , (c',\hat{v}')) \in E$ we have $c = c'$.

If we solve the PG that is the unification of a VPG we have solved the VPG, as shown in the next theorem.
\begin{theorem}
	\label{theA_solve_UVPG_is_solve_VPG}
	Given 
	\begin{itemize}
		\item VPG $\hat{G} =  (\hat{V},\hat{V}_0,\hat{V}_1, \hat{E},\hat{\Omega}, \mathfrak{C},\theta)$,
		\item some configuration $c \in \mathfrak{C}$,
		\item winning sets $\hat{W}^c_0$ and $\hat{W}^c_1$ for game $\hat{G}$ and
		\item winning sets $W_0$ and $W_1$ for game $\hat{G}_{\downarrow}$
	\end{itemize}
it holds that
	\[(c,\hat{v}) \in W_\alpha \iff \hat{v} \in \hat{W}^c_\alpha  \text{, for }\alpha \in \{0,1\}  \]
	\begin{proof}
		The bi-implication is equal to  the following to implications.
		\[ (c,\hat{v}) \in W_\alpha \implies \hat{v} \in \hat{W}^c_\alpha  \text{, for }\alpha \in \{0,1\} \]
		and
		\[ (c,\hat{v}) \notin W_\alpha\implies \hat{v} \notin \hat{W}^c_\alpha \text{, for }\alpha \in \{0,1\}  \]
		
		Since the winning sets partition the game we have $\hat{v} \notin \hat{W}^c_\alpha \implies \hat{v} \in \hat{W}^c_{\overline{\alpha}}$ (similar for set $W$). Therefore it is sufficient to prove only the first implication.
		
		Let $(c,\hat{v}) \in W_\alpha$, player $\alpha$ has a strategy to win game $\hat{G}_{\downarrow}$ from vertex $(c,\hat{v})$. Since $\hat{G}_{\downarrow}$ is the union of all the projections of $\hat{G}$ we can apply the same strategy to game $\hat{G}_{|c}$ to win vertex $\hat{v}$ as player $\alpha$. Because we can win $\hat{v}$ in the projection of $\hat{G}$ to $c$ we have $\hat{v} \in \hat{W}^c_\alpha$.
	\end{proof}
\end{theorem}

\subsection{Representing unified parity games}
Unified PGs have a specific structure because they are the union of PGs that have the same vertices with the same owner and priority. Because they have the same priority we don't actually need to create a new function that is the unification of all the projections, we can simply use the original priority assignment function because the following relation holds:
\[ \Omega(c,\hat{v}) = \hat{\Omega}(\hat{v}) \]
Similarly we can use the original partition sets $\hat{V}_0$ and $\hat{V}_1$ instead of having the new partition $V_0$ and $V_1$ because the following relations holds:
\[ (c,\hat{v}) \in V_0 \iff \hat{v}\in \hat{V}_0 \]
\[ (c,\hat{v}) \in V_1 \iff \hat{v}\in \hat{V}_1 \]
So instead of considering unified PG $(V,V_0,V_1,E,\Omega)$ we will consider $(V,\hat{V}_0,\hat{V}_1,E,\hat{\Omega})$. 

Next we consider how we represent vertices and edges in a unified PG. A set $X \subseteq (\mathfrak{C} \times \hat{V})$ can be represented as a complete function $f : \hat{V} \rightarrow 2^\mathfrak{C}$. The set $X$ and function $f$ are equivalent, denoted by the operator $=_\lambda$, iff the following relation holds:
\[ (c,\hat{v}) \in X \iff c \in f(\hat{v}) \]
We can also represent edges as a complete function $f : \hat{E} \rightarrow 2^\mathfrak{C}$. The set $E$ and function $f$ are equivalent, denoted by the operator $=_\lambda$, iff the following relation holds:
\[ ((c,\hat{v}),(c,\hat{v}')) \in E \iff c \in f(\hat{v},\hat{v}') \]
We write $\lambda^\emptyset$ to denote the function that maps every element to $\emptyset$, clearly $\lambda^\emptyset =_\lambda \emptyset$. We call using a set of pairs to represent vertices and edges a \textit{set-wise} representation and using functions a \textit{function-wise} representation.

\subsection{Projections and totality}
A unified PG can be projected back to one of the games from which it is the union.
\begin{definition}
	The projection of unified PG $G = (V,\hat{V}_0, \hat{V}_1,E,\hat{\Omega})$ to configuration $c$, denoted as $G_{|c}$, is the parity game $(V',\hat{V}_0,\hat{V}_1,E',\hat{\Omega})$ such that $V' = \{ \hat{v}\ |\ (c,\hat{v}) \in V \}$ and $E' = \{ (\hat{v},\hat{w})\ |\ ((c,\hat{v}),(c,\hat{w})) \in E \} $.
\end{definition}

One of the properties of a PG is its totality; a game is total if every vertex has at least 1 outgoing vertex. VPGs are also total, meaning that every vertex has, for every configuration $c \in \mathfrak{C}$, at least 1 outgoing vertex admitting $c$. Because VPGs are total their unifications are also total. Since edges in a unified PG don't cross configurations we can conclude that a unified PG is total iff every projection is total.