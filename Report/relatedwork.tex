Prior work has been done to verify SPLs, we discuss four notable contributions.

In \cite{CheckingLotsOfSystems} a method is introduced to verify for which products in an FTS an LTL property holds. It does so by constructing a B{\"u}chi automaton representing the complement of the LTL property and checking if the synchronous product of the automaton and the transition system has an empty language \cite{LTLusingAutomata}. When applying this method to verify LTSs the reachability of the synchronous product is explored. For FTSs the paper introduces a reachability definition that determines for every product if a state is reachable. It is observed that a symbolic representation of the sets of products is advantageous when keeping track of the reachability. The paper uses the minepump example \cite{Kramer1983CONICAI} to perform an experimental evaluation and find a substantial gain using verifying a family of products collectively as opposed to independently.

This work is expanded upon in \cite{FTSLTL}. The performance of such an approach is further elaborated upon and it is confirmed that a collective approach indeed outperforms an independent approach. Furthermore an extension of LTL is presented, namely featured LTL (fLTL). fLTL parametrizes LTL to be able to express properties in terms of features. Using this language one can distinguish between products when expressing temporal requirements.

In \cite{FTSCTL} symbolic model-checking is used to verify SPLs. fCTL is introduced as an extension of CTL that is able to reason about features. Verification of a CTL property can be done by expressing the CTL property as a tree, its parse tree, and doing a bottom-up traversal of it, deciding at every node what states satisfy the subformula. The paper proposes a way to symbolically represents FTSs, introduces a parse tree definition for the fCTL language and introduces an algorithm to do a bottom-up traversal of a fCTL parse tree, deciding at every node which state-product pairs satisfy it. These concepts are put in practice using the symbolic model checking toolset NuSMV \cite{NuSMV} as a basis and extend the language to express variability. The paper uses the elevator example \cite{PLATH200153}, modified to have 9 features, to show that symbolically model checking an SPL collectively using the methods proposed can significantly improve the performance compared to symbolically model checking all products independently.

Finally, in \cite{FTSMu} an extension of the modal $\mu$-calculus, namely $\mu L_f$, is proposed that can reason about features. In \cite{FamBasedModelCheckingWithMCRL2} it is shown how properties expressed in $\mu L_f$ can be embedded in first order $\mu$-calculus and how the mCRL2 toolset \cite{mCRL2Toolset} can be put to work to verify these properties. An algorithm is proposed that recursively partitions the products based on their features. After every partitioning the algorithm checks if the remaining set of products all satisfy the requirement or none of them satisfy the requirement. If either is true then that recursion is done, otherwise the algorithm continues. It is observed that the performance of this approach depends largely on deciding how to partition the sets of products. What would be a good heuristic/approach to splitting products is left unanswered in the paper.