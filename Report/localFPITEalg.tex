The incremental pre-solve algorithm is particularly appropriate for local solving; if we find ${v}_0$ in  $P_\alpha$ then we know that ${v}_0$ is won by player $\alpha$ for every configuration, therefore we are done for that particular recursion. This can potentially reduce the recursion depth of the algorithm and therefore reduce the number of (pessimistic) games solved.

Furthermore, when there is only a single configuration left the incremental pre-solve algorithm solves the corresponding parity game. When taking a local approach it is sufficient to solve this parity game locally using the local fixed-point iteration algorithm. Note that the pessimistic games still must be solved globally to find as much assistance as possible for further recursions.

If $v_0$ is not found in either $P_0$ or $P_1$ and is only solved when there is one configuration left, then the local algorithm behaves the same as the global algorithm; we have identical worst-case time complexities: $O(c*e*n^d)$.