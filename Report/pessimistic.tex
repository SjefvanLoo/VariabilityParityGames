Given a VPG with configurations $\mathfrak{C}$ we can try to determine sets $P_0,P_1$ such that the vertices in set $P_\alpha$ are won by player $\alpha \in \{0,1\}$ for any configuration in $\mathfrak{C}$. We can do so by creating a \textit{pessimistic} PG; a pessimistic PG is a parity game created from a VPG for a player $\alpha \in \{0,1\}$ such that the PG allows all edges that player $\overline{\alpha}$ might take but only allows edges for $\alpha$ when that edge admits all the configurations in $\mathfrak{C}$.
\begin{definition}
	\label{def_pess_game}
	Given VPG $G = (V,V_0,V_1,E,\Omega, \mathfrak{C},\theta)$, we can create pessimistic PG $G_{\triangleright\alpha}$ for player $\alpha \in \{0,1\}$. We have	
	\[ G_{\triangleright\alpha} = \{V,V_0,V_1,E',\Omega \} \]
	such that
	\[ E' = \{ (v,w) \in E\ |\ v \in V_{\overline{\alpha}} \vee \theta(v,w) = \mathfrak{C} \} \]
\end{definition}


Note that pessimistic parity games are not necessarily total. A play in a PG that is not total might result in a finite path, in such a case the player that can't make a move looses the play.

When solving a pessimistic PG $G_{\triangleright\alpha}$ we get winning sets $W_0,W_1$, every vertex in $W_\alpha$ is winning for player $\alpha$ in $G$ played for any configuration, as shown in the following theorem.
\begin{theorem}
	\label{the_pess_is_winning_for_all_conf}
	Given:
	\begin{itemize}
		\item VPG $G = (V,V_0,V_1,E,\Omega,\mathfrak{C},\theta)$,
		\item configuration $c \in \mathfrak{C}$,
		\item winning sets $W_0^c, W_1^c$ for game $G$,
		\item player $\alpha \in \{0,1\}$ and
		\item pessimistic PG $G_{\triangleright\alpha}$ with winning sets $P_0$ and $P_1$
	\end{itemize}
we have $P_\alpha \subseteq W_\alpha^c$.
	\begin{proof}
		Player $\alpha$ has a strategy in game $G_{\triangleright\alpha}$ such that vertices in $P_\alpha$ are won. We will show that this strategy can also be applied to game $G_{|c}$ to win the same or more vertices.
		
		First we observe that any edge that is taken by player $\alpha$ in game $G_{\triangleright\alpha}$ can also be taken in game $G_{|c}$ so player $\alpha$ can play the same strategy in game $G_{|c}$.
		
		For player $\overline{\alpha}$ there are possibly edges that can be taken in $G_{\triangleright\alpha}$ but can't be taken in $G_{|c}$, in such a case player $\overline{\alpha}$'s choices are limited in game $G_{|c}$ compared to $G_{\triangleright\alpha}$ so if player $\overline{\alpha}$ can't win a vertex in $G_{\triangleright\alpha}$ then he/she can't win that vertex in $G_{|c}$.
		
		We can conclude that applying the strategy from game $G_{\triangleright\alpha}$ in game $G_{|c}$ for player $\alpha$ wins the same or more vertices.
	\end{proof}
\end{theorem}

\subsection{Configuration partitioning}
\begin{definition}
	Given VPG $G = (V,V_0,V_1,E,\Omega,\mathfrak{C},\theta)$ and non-empty set $\mathfrak{X} \subseteq \mathfrak{C}$ we define the subgame $G \cap \mathfrak{X} = (V,V_0,V_1,E',\Omega,\mathfrak{C}', \theta')$ such that
	\begin{itemize}
		\item $\mathfrak{C}' =\mathfrak{C} \cap \mathfrak{X}$,
		\item $\theta'(e) = \theta(e) \cap \mathfrak{C}'$ and
		\item $E' = \{ e \in E\ |\ \theta'(e) \neq \emptyset \}$.
	\end{itemize}
\end{definition}
VPGs are total, meaning that for every configuration and every vertex there is an outgoing edge from that vertex admitting that configuration. In subgames the set of configurations is restricted and only edge guards and edges are removed for configurations that fall outside the restricted set, therefore we still have totality.

Furthermore it is trivial to see that every projection $G_{|c}$ is equal to $(G \cap \mathfrak{X})_{|c}$ for any $c \in \mathfrak{C} \cap \mathfrak{X}$.

Finally the subset operator is associative, meaning $(G \cap \mathfrak{X}) \cap \mathfrak{X}' = G \cap (\mathfrak{X} \cap \mathfrak{X}') = G \cap \mathfrak{X} \cap \mathfrak{X}'$.

Vertices in winning set $P_\alpha$ for $G_{\triangleright\alpha}$ are also winning for player $\alpha$ in pessimistic subgames of $G$, as shown in the following lemma.
\begin{lemma}
	\label{lem_pessimistic_subgames}
	Given:
	\begin{itemize}
		\item VPG $G = (V,V_0,V_1,E,\Omega, \mathfrak{C},\theta)$,
		\item $P_0$ being the winning set of game $G_{\triangleright0}$ for player $0$,
		\item $P_1$ being the winning set of game $G_{\triangleright1}$ for player $1$,
		\item non-empty set $\mathfrak{X} \subseteq \mathfrak{C}$,
		\item player $\alpha \in \{0,1\}$ and
		\item winning sets $Q_0,Q_1$ for game $(G \cap \mathfrak{X})_{\triangleright\alpha}$
	\end{itemize}
	we have
	\[ P_0 \subseteq Q_0 \]
	\[ P_1 \subseteq Q_1 \]
	\begin{proof}
	
		Let edge $(v,w)$ be an edge in game $G_{\triangleright\alpha}$ with $v \in V_\alpha$. Edge $(v,w)$ admits all configuration in $\mathfrak{C}$ so it also admits all configuration in $\mathfrak{C} \cap \mathfrak{X}$, therefore we can conclude that edge $(v,w)$ is also an edge of game $(G\cap \mathfrak{X})_{\triangleright\alpha}$.
		
		Let edge $(v,w)$ be an edge in game $(G \cap \mathfrak{X})_{\triangleright\alpha}$ with $v \in V_{\overline{\alpha}}$. The edge admits some configuration in $\mathfrak{C} \cap \mathfrak{X}$, this configuration is also in $\mathfrak{C}$ so we can conclude that edge $(v,w)$ is also an edge of game $G_{\triangleright\alpha}$.
		
		We have concluded that game $(G \cap \mathfrak{X})_{\triangleright\alpha}$ has the same or more edges for player $\alpha$ as game $G_{\triangleright\alpha}$ and the same or less edges for player $\overline{\alpha}$. Therefore we can conclude that any vertex won by player $\alpha$ in $G_{\triangleright\alpha}$ is also won by $\alpha$ in game $(G \cap \mathfrak{X})_{\triangleright\alpha}$, ie. $P_\alpha \subseteq Q_\alpha$.
		
		
		Let $v \in P_{\overline{\alpha}}$, using theorem \ref{the_pess_is_winning_for_all_conf} we find that $v$ is winning for player $\overline{\alpha}$ in $G_{|c}$ for any $c \in \mathfrak{C}$. Because projections of subgames are the same as projections of the original game we can conclude that $v$ is winning for player $\overline{\alpha}$ in $(G \cap \mathfrak{X})_{|c}$ for any $c \in \mathfrak{C} \cap \mathfrak{X}$.
		
		Assume $v \notin Q_{\overline{\alpha}}$ then $v \in Q_{\alpha}$ and using theorem \ref{the_pess_is_winning_for_all_conf} we find that $v$ is winning for player $\alpha$ in $(G \cap \mathfrak{X})_{|c}$ for any $c \in \mathfrak{C} \cap \mathfrak{X}$. This is a contradiction so we can conclude $v \in Q_{\overline{\alpha}}$ and therefore $P_{\overline{\alpha}} \subseteq Q_{\overline{\alpha}}$.
	\end{proof}
\end{lemma}
